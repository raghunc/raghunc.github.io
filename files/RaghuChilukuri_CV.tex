\documentclass[10pt]{article}
\usepackage[margin=0.8in]{geometry}
%\usepackage{enumitem}
\usepackage{amsmath}
%\usepackage{fullpage}
\usepackage{amssymb}
\usepackage{makeidx}
\usepackage{hyperref}
%\usepackage{hyperlatex}
%\usepackage{tgpagella}
%\usepackage{lmodern}
\usepackage[T1]{fontenc}
\pagestyle{empty}

\newcommand{\CPP}
{C\nolinebreak[4]\hspace{-.05em}\raisebox{.22ex}{\footnotesize\bf ++}}
%\voffset = 2pt

\begin{document}

\begin{center}
 \textbf{ \textsc{\Large Raghu Nandan Chilukuri}}

%\noindent \textbf{Raghu Nandan Chilukuri} \hfill \href{https://sites.google.com/site/raghunchilukuri/home}{https://sites.google.com/site/raghunchilukuri/home}\newline
% 688 Riddle Rd. Apt. 1100J  \textbullet \  Cincinnati, OH 45220
%\\
%1(513)477-5190  \textbullet \ chilukrn@mail.uc.edu \  \textbullet \ raghu.rnc@gmail.com
\href{mailto:chilukrn@mail.uc.edu}{chilukrn@mail.uc.edu} $\bullet$ \href{raghu.rnc@gmail.com}{raghu.rnc@gmail.com}
 \\
 %\href{http://homepages.uc.edu/\~{}chilukrn/}{http://homepages.uc.edu/~chilukrn/}
\href{http://raghunc.org}{raghunc.org}
\end{center}

			       
%\subsection*{Objective:}
%To work in a professional environment in an organization that fosters learning and utilizes my skills in working on cutting edge technologies towards achieving organizational goals.
%\subsection*{Areas of Interest:}
 %VLSI design, Embedded systems.

\subsection*{\textsc{\large Education}}                 
\textbf{University of Cincinnati} \hfill \textbf{Cincinnati,OH,USA} \newline
M.S.(Electrical Engineering) \hfill 2011 - 2014 \newline
%GPA: 3.34/4 \newline
%\textit{Research Interests: Spintronics, Nanoelectronics, Condensed Matter Physics} \newline
     
\noindent \textbf{Birla Institute of Technology and Science(BITS)-Pilani} (Goa campus) \hfill  \textbf{Goa,India} \newline %\href{http://www.bits-goa.ac.in}{www.bits-goa.ac.in} \newline                                        
B.Eng.(Hons) (Electrical \& Electronics Engineering) \hfill 2006 - 2010 \newline
%GPA: 8.06/10 \newline
%\textit{Research Interests: Quantum Computing, Nanoelectronics }

%\subsection*{\textsc{Test Scores}:}
%\textbf{GRE General Test}: 1390 (V:600, Q:790, AWA: 4.0)\newline
%\textbf{TOEFL iBT}: 105/120

\subsection*{\textsc{\large Research}}
\textbf{Spintronics and Vacuum Nanoelectronics Laboratory} \hfill \textbf{2012 - 2014}
\begin{itemize}
\item Numerical simulation of quantum random walks and modelling noisy quantum walks. (Advisor: Prof. Marc Cahay, University of Cincinnati).
\end{itemize}

\subsection*{\textsc{\large Teaching Experience}}
\textbf{Athletic Tutoring} \hfill \textbf{Fall 2012}
\begin{itemize}
\item Tutored introductory physics for student athletes at the University of Cincinnati.
\end{itemize}

\subsection*{\textsc{\large Work Experience}}
\noindent \textbf{Intel Corporation} \hfill \textbf{Bangalore, India} \newline
\textit{Component Design Engineer} \hfill March 2016-present
\begin{itemize}
\item Worked in Front-End (FE) verification team, and the support and methodology owner for VCS, Certitude (both Synopsys), and Intel specific internal tools.
\item Scripting support for internal EDA tools of Intel.
\end{itemize}

\noindent \textbf{JustDial India Private Ltd.} \hfill \textbf{Bangalore, India} \newline
\textit{Data Scientist} \hfill September 2015-February 2016
\begin{itemize}
\item Worked on implementing a recommender system for the e-commerce shopfront of JustDial
\item Implemented features using custom versions of clustering algorithms
\end{itemize}

\noindent \textbf{Cambridge Silicon Radio(CSR) India Private Ltd.} \hfill \textbf{Bangalore, India} \newline
\textit{Firmware test Intern} \hfill February 2011-August 2011
\begin{itemize}
\item Characterizing physical parameters such as power, gain and stress-test of bluetooth chips.
\item Testing the Bluetooth lower stack firmware for these new chips.
\item Designed and implemented an audio test system to test different profiles for audio streaming and BLE (Bluetooth low energy) file transfer between multiple devices.
\end{itemize}

\noindent \textbf{Dept. of Electrical Engineering,IIT-Delhi} \hfill \textbf{Delhi, India}  \newline
\textit{Change Detection Algorithms} \hfill October 2010-January 2011
\begin{itemize}
\item Change detection algorithms using segmentation and defocussing.
\item Implemented using open-source computer vision libraries on a smart camera system.
\end{itemize}

\noindent \textbf{Infinera India Private Ltd.} \hfill \textbf{Bangalore, India} \newline
\textit{System Verification} \hfill January 2010-June 2010
\begin{itemize}
\item Board simulation and system level testing of a system of line cards.
\item Implemented a basic Verilog parser for netlist parsing and verification.
\end{itemize}

\subsection*{\textsc{\large Summer Intern}}
\textbf{Vikram Sarabhai Space Center (VSSC)} \hfill \textbf{Thiruvananthapuram, India} \newline 
\textit{Testing of control module electronics} \hfill May 2008-July 2008
\begin{itemize}
\item Interfaced and programmed a PIC microcontroller to generate waveforms for testing control electronics module and mechanical characteristics of an electro-mechanical system.
\end{itemize}

\subsection*{\textsc{\large Academic Projects}}
\begin{itemize}
\item Noisy Quantum walks: Working on an independent project on quantum-operator representation of noisy quantum walks in order to generalize noisy discrete quantum walk with Prof. R. Srikanth, PPISR, India (\textit{2016})
\item Developed a routing tool (in C++) based on Channel Routing. The tool also produces a layout file for viewing with Magic layout editor. (\textit{2012})
\item Implemented a variant of Kerninghan-Lin graph partition algorithm in C.(\textit{2012})
\item Quantum Error Correction: Quantum error correction methods and a comparison of classical error correction techniques with their quantum counterparts. This project discusses few quantum error correction techniques as an extension of their classical counterparts. (\textit{2009})
\item Selected Topics in Quantum Optics-Generation of Coherent States and Squeezed States : A study of semi-classical and non-classical optical states and their generation. (\textit{2009})
\end{itemize}

\subsection*{\textsc{\large Papers/Technical Reports}}
\begin{itemize}
\item Dr.A.K.Biswas, Ch.Raghu Nandan, V.Jayanth,``A sphere moving down the surface of a static sphere and a simple phase diagram''. (arXiv classical physics: \url{http://arxiv.org/abs/0808.3531v2})

\item ``Design of a virtual Hawk-Eye system using LabVIEW'' : A project to simulate the 3 dimensional motion of a projectile (a tennis ball in this case). This was submitted to 'VI Mantra 2009' contest by National Instruments, India, as a paper with the same name.
 \end{itemize}

\subsection*{\textsc{\large Graduate Courses}}
Semiconductor microfabrication\hfill Fundamentals of MEMS \newline 
Electromagnetic Theory \hfill Characterization of materials by optical methods \newline
Quantum Mechanics \hfill Quantum Computation \newline
Semiconductor Physics \hfill Advanced Solid State Physics(Many body theory-Green's function formalism)

\subsection*{\textsc{\large Technical Skills}}
\textbf{Languages:} C, \CPP , Julia, Fortran, Verilog, Perl, Python, Assembly, \LaTeX. \newline
\textbf{Software Packages:} Matlab/Octave, Pspice, Xilinx ISE, Altera Quartus, Magic, LabVIEW \newline
\textbf{Operating systems:} GNU/Linux, Windows


\subsection*{\textsc{\large Academic Achievements}}
\begin{itemize}
\item Recipient of University Graduate Scholarship(UGS) at University of Cincinnati.
\item Selected for Indian National Chemistry Olympiad (INChO), 2006 (top 1\% among an estimated 20,000 aspirants who appeared for National Standard Exam in Chemistry- NSEC).
\item Recipient of Merit cum Need scholarship of Bits-Pilani, Goa campus for six semesters.
\end{itemize}

\end{document}
