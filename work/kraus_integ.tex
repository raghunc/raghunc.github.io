\documentclass[english]{article}

\usepackage{babel}
\usepackage{graphicx}
\usepackage{amsmath}
\usepackage{amssymb}
\usepackage[margin=0.75in]{geometry}
\usepackage{mathtools}
\DeclarePairedDelimiter\abs{\lvert}{\rvert}
\DeclarePairedDelimiter\norm{\lVert}{\rVert}
%\DeclarePairedDelimiter\abs*{\lvert*}{\rvert*}
%\DeclarePairedDelimiter\norm*{\lVert*}{\rVert*}

\usepackage[T1]{fontenc}





\begin{document}
\title{Kraus operator representation of the noisy quantum walk}
\maketitle

\section{Quantum walk in Kraus representation}

\subsection{Kraus representation}

Now we try the Kraus operator representation of noisy quantum walk. Here, the ``coin'' operator is the source of the noise. Assume the initial coin state (or spin state) is:

$$ \mid\!s\!\rangle = a_{\uparrow} \mid\!\uparrow\!\rangle + a_{\downarrow} \mid\!\downarrow\!\rangle $$

The corresponding density matrix would look:

\begin{eqnarray}
  \rho_{0} = \mid\!s\!\rangle \langle\!s\!\mid = \left[\begin{array}{c} a_{\uparrow}\\ a_{\downarrow}\end{array}\right] \left[\begin{array}{c}a_{\uparrow}^{*}\quad a_{\downarrow}^{*}\end{array} \right]
\rho_{0}  = \left[\begin{array}{cc}\abs{a_{\uparrow}}^{2} & a_{\uparrow}a_{\downarrow}^{*} \\
               a_{\uparrow}^{*}a_{\downarrow} & \abs{a_{\downarrow}}^{2}
    \end{array} \right]
\end{eqnarray}


The noisy quantum operator acts on the initial state to give the final state as:
$$
\left[\begin{array}{cc} r & \sqrt{1 - r^{2}} \\\sqrt{1 - r^{2}} & -r \end{array}\right] \left[\begin{array}{c} a_{\uparrow} \\ a_{\downarrow} \end{array}\right] = \left[\begin{array}{c}ra_{\uparrow} + a_{\downarrow}\sqrt{1 - r^{2}} \\ a_{\uparrow}\sqrt{1 - r^{2}} - ra_{\downarrow} \end{array}\right]
$$

The density of the final state in the above equation can be written:
\begin{equation}\label{eq:finalstate-densitymatrix}
\begin{aligned}
\rho^{\prime} & = {} \left[\begin{array}{c}ra_{\uparrow} + a_{\downarrow}\sqrt{1 - r^{2}} \\ a_{\uparrow}\sqrt{1 - r^{2}} - ra_{\downarrow} \end{array}\right] \left[\begin{array}{cc}ra_{\uparrow}^{*} + a_{\downarrow}^{*}\sqrt{1 - r^{2}} & a_{\uparrow}^{*}\sqrt{1 - r^{2}} - ra_{\downarrow}^{*} \end{array}\right] \\
& = \left[\begin{array}{cc} r^{2}\abs{a_{\uparrow}}^2+(1-r^{2})\abs{a_{\downarrow}}^{2}+(r\sqrt{1-r^{2}})(a_{\uparrow}a_{\downarrow}^{*}+a_{\uparrow}^{*}a_{\downarrow}) & (r\sqrt{1-r^{2}})(\abs{a_{\uparrow}}^{2} - \abs{a_{\downarrow}}^{2})+(1-r^{2})a_{\downarrow}a_{\uparrow}^{*}-r^{2}a_{\uparrow}a_{\downarrow}^{*} \\
(r\sqrt{1-r^{2}})(\abs{a_{\uparrow}}^{2} - \abs{a_{\downarrow}}^{2})+(1-r^{2})a_{\uparrow}a_{\downarrow}^{*}-r^{2}a_{\downarrow}a_{\uparrow}^{*} & (1-r^{2})\abs{a_{\uparrow}}+ r^{2}\abs{a_{\uparrow}}^2-(r\sqrt{1-r^{2}})(a_{\uparrow}a_{\downarrow}^{*}+a_{\uparrow}^{*}a_{\downarrow})
  \end{array}\right]
\end{aligned}
\end{equation}

%We have tried to represent the density matrix $\rho^{\prime}$ given in \eqref{eq:finalstate-densitymatrix} in Kraus representatiion.
This can be separated into 3 matrices as:

\begin{equation}
  \begin{aligned}
  \rho^{\prime} = {} & r^{2} \left[\begin{array}{cc}\abs{a_{\uparrow}}^{2} & -a_{\uparrow}a_{\downarrow}^{*} \\
               -a_{\uparrow}^{*}a_{\downarrow} & \abs{a_{\downarrow}}^{2} \end{array} \right] \\
  & + (1-r^{2})\left[\begin{array}{cc}\abs{a_{\downarrow}}^{2} & a_{\downarrow}a_{\uparrow}^{*} \\
               a_{\uparrow}a_{\downarrow}^{*} & \abs{a_{\uparrow}}^{2} \end{array} \right] \\
  & + (r\sqrt{1-r^{2}})\left[\begin{array}{cc}a_{\uparrow}a_{\downarrow}^{*}+a_{\downarrow}a_{\uparrow}^{*} & (\abs{a_{\uparrow}}^{2}-\abs{a_{\downarrow}}^{2}) \\
             (\abs{a_{\uparrow}}^{2}-\abs{a_{\downarrow}}^{2}) & -(a_{\uparrow}a_{\downarrow}^{*}+a_{\downarrow}a_{\uparrow}^{*})\end{array} \right]
  \end{aligned}
\end{equation}

If the limits of $r$ are from $-p$ to $+p$, it can be written in Kraus form as:

\begin{equation}\label{eq:Krausform-integral}
  \begin{aligned}
    \rho_{s} = \int\limits_{r=-p}^{+p}\rho^{\prime}\,\mathrm{d}r = {} & \int\limits_{r=-p}^{+p}\mathrm{d}r\,r^{2}Z\rho_{0}\!Z^{\dagger} + \int\limits_{r=-p}^{+p}\mathrm{d}r\,(1-r^{2})X\rho_{0}\!X^{\dagger} \\
                  & + \int\limits_{r=-p}^{+p}\mathrm{d}r\,(r\sqrt{1-r^{2}})\left[\begin{array}{cc}a_{\uparrow}a_{\downarrow}^{*}+a_{\downarrow}a_{\uparrow}^{*} & (\abs{a_{\uparrow}}^{2}-\abs{a_{\downarrow}}^{2}) \\
             (\abs{a_{\uparrow}}^{2}-\abs{a_{\downarrow}}^{2}) & -(a_{\uparrow}a_{\downarrow}^{*}+a_{\downarrow}a_{\uparrow}^{*} \end{array}\right]
  \end{aligned}
\end{equation}

The terms $Z\rho\!Z^{\dagger}$ and $X\rho\!X^{\dagger}$, and the last matrix are independent of ``$r$'' and hence donot contribute to the integral, and the final state $\rho_{s}$ can be written as:

\begin{equation}
  \begin{aligned}
    \rho_{s} = \int\limits_{r=-p}^{+p}\rho^{\prime}\,\mathrm{d}r = {} & \left.\left(\dfrac{r^{3}}{3}\right)\right|_{-p}^{+p}Z\rho_{0}\!Z^{\dagger} + \left.\left(r -\dfrac{r^{3}}{3}\right)\right|_{-p}^{+p}X\rho_{0}\!X^{\dagger} \\
                  & + \left.\left(-\dfrac{1}{3}\right)\left(1-r^{2}\right)^{\frac{3}{2}}\right|_{-p}^{+p}\left[\begin{array}{cc}a_{\uparrow}a_{\downarrow}^{*}+a_{\downarrow}a_{\uparrow}^{*} & (\abs{a_{\uparrow}}^{2}-\abs{a_{\downarrow}}^{2}) \\
             (\abs{a_{\uparrow}}^{2}-\abs{a_{\downarrow}}^{2}) & -(a_{\uparrow}a_{\downarrow}^{*}+a_{\downarrow}a_{\uparrow}^{*} \end{array}\right]
    \end{aligned}
\end{equation}

If we let $r$ range from $-1$ to $+1$, the last term gets cancelled out, and the final form looks:
\begin{equation}
  %\begin{aligned}
    \rho_{s} = \left(\dfrac{2}{3}\right)Z\rho_{0}\!Z^{\dagger} + \left(\dfrac{4}{3}\right)X\rho_{0}\!X^{\dagger}
    %\end{aligned}
\end{equation}


However, earlier, I was assuming that $r$ ranges from $0$ to $1$, and under this assumption, eq.\eqref{eq:Krausform-integral} changes to:
\begin{equation}\label{eq:Krausform-integral-actual}
  \begin{aligned}
    \rho_{s} = \int\limits_{r=-0}^{1}\rho^{\prime}\,\mathrm{d}r = {} & \int\limits_{r=0}^{1}\mathrm{d}r\,r^{2}Z\rho_{0}\!Z^{\dagger} + \int\limits_{r=0}^{1}\mathrm{d}r\,(1-r^{2})X\rho_{0}\!X^{\dagger} \\
                  & + \int\limits_{r=0}^{1}\mathrm{d}r\,(r\sqrt{1-r^{2}})\left[\begin{array}{cc}a_{\uparrow}a_{\downarrow}^{*}+a_{\downarrow}a_{\uparrow}^{*} & (\abs{a_{\uparrow}}^{2}-\abs{a_{\downarrow}}^{2}) \\
             (\abs{a_{\uparrow}}^{2}-\abs{a_{\downarrow}}^{2}) & -(a_{\uparrow}a_{\downarrow}^{*}+a_{\downarrow}a_{\uparrow}^{*} \end{array}\right]
  \end{aligned}
\end{equation}
which can simplified to give:
\begin{equation}\label{eq:actual-decomp}
  \begin{aligned}
    \rho_{s} = {} & \left(\dfrac{1}{3}\right)Z\rho_{0}\!Z^{\dagger} + \left(\dfrac{2}{3}\right)X\rho_{0}\!X^{\dagger} \\
                  & + \left(\dfrac{1}{3}\right)\left[\begin{array}{cc}a_{\uparrow}a_{\downarrow}^{*}+a_{\downarrow}a_{\uparrow}^{*} & (\abs{a_{\uparrow}}^{2}-\abs{a_{\downarrow}}^{2}) \\
             (\abs{a_{\uparrow}}^{2}-\abs{a_{\downarrow}}^{2}) & -(a_{\uparrow}a_{\downarrow}^{*}+a_{\downarrow}a_{\uparrow}^{*} \end{array}\right]
  \end{aligned}
\end{equation}
wherein the last matrix doesn't vanish.

%It can be seen that the first two terms in the above equation are in the Kraus form, whereas the last one couldn't be cast in that form. I have tried combinations of $X$, $Y$, $Z$ (Pauli matrices), and $I$ but it couldn't be cast in that form: the best was some thing of the form $Z\rho\!I + I\rho\!Z$.

\end{document}
